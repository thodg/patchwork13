\documentclass[14pt,a4paper]{report}
\usepackage[francais]{babel}
\usepackage[T1]{fontenc}
\usepackage{graphicx}

\begin{document}

\begin{center}
\rule{0cm}{19mm} \\

{\Huge \textbf{Rapport de Soutenance n$^o$1}} \\[23mm]

{\huge 2B2S} \\[8mm]

Thomas 'billich' De Grivel \\
Maxime 'loucha\_m' Louchart \\
Bruno 'Broen' Malaquin \\
Julien 'Splin' Valentin \\[32mm]

\centerline{{\includegraphics{patchwork13.jpg}}}

\end{center}

\newpage

\tableofcontents

\newpage



\chapter{Introduction}

Voici donc venue la premi\`ere soutenance du projet Patchwork13!

Cet outil de synthe\`ese modulaire coupl\'e avec un cluster qui a pour
but ultime d'offrir \`a un utilisateur lambda la possibilit\'e d'effectuer
un nombre irtuellement illimit\'e de fonctions complexes
\'eventuellement sans avoir
la moindre notion de programmation.

Le projet reprend un projet open source sous license GNU qui a
d\'emarr\'e l'ann\'ee derni\`ere. Il n'\'etait encore qu'a
l'\'etat d'embryon lorsque nous avons repris le d\'eveloppement
du projet. \\

Bien que la plupart des fonctions de base du noyau \'etaient
d\'ej\`a fonctionnelles, le projet ne pr\'esentait aucun
signe de fonctionnement visible a part
un executable en mode console affichant une \'etrange serie de
nombres (un sinus ?) et une compilation sans erreurs\ldots \\

Les choses ont bien chang\'ees depuis puisque le projet s'est
vu dot\'e de deux aspects compl\`etement nouveaux
(bien que pr\'evus) : \\

Nous avons d\'esormais une {\bf interface graphique} qui permettra d'utiliser
la librairie gr\^ace \`a la souris. Elle pr\'esente d\'eja plusieur
fen\^etres et permet de voir les patches (plug-ins) install\'es sur
le syst\^eme sous forme d'arborescence. \\

Ind\'ependemment de cela, un syst\`eme de {\bf cluster} a \'et\'e
largement entam\'e et dispose d\'ej\`a de fonctions r\'eseau comme
la d\'etection des serveurs disponibles sur le r\'eseau local
et l'envoi de messages texte.

Ce cluster permettra d'avoir en toute simplicit\'e la puissance de
centaines de machines \`a la port\'ee d'une banale souris et d'un
esprit cr\'eatif, m\^eme s celui-ci n'a aucune id\'ee du fonctionnement
des machines.\\

Nous vous pr\'esentons plus en d\'etail dans les pages qui suivent
l'ensemble de ce que nous avons realis\'e depuis le 18 novembre,
ainsi que le travail pr\'evu pour la prochaine
soutenance, et ce pour chaque partie du projet.

Commen\c{c}ons par la plus ancienne, qui se cache tout juste
deri\`ere cette page.


\chapter{Le noyau ({\bf pw13})}

\section{Ce qui a \'et\'e fait}

Toutes les fonctions du noyau (une trentaine) ont \'et\'e
renomm\'ees pour respecter les normes
des librairies : tous les symboles export\'es ont \'et\'es pre\'efix\'es
par ``{\tt pw13\_}'' pour ne pas faire de conflit avec les programmes qui
utiliseront la librairie. \\

La portabilit\'e sous Windows est d\'esormais compl\`ete, le noyau se
compile avec {\tt gcc} sous MinGW/MSYS (et non cygwin) ce qui donne donc du
code windows {\bf natif} (pas de librarie ou de license suppl\'ementaire
\`a inclure avec le programme).

Nous obtenons donc actuellemnt une DLL parfaitement utilisable sous Win32,
ce qui est une op\'eration autrement plus complexe que de compiler
une librairie sous UNIX, et qui pose des
contraintes tr\`es strictes sur la mani\`ere de compiler. \\

Le d\'eveloppment ayant d\'emarr\'e sous Gentoo/Linux, le portage
sous Debian et BSD n'a pos\'e aucun probl\`eme, et
MacOS X est totalement support\'e lorsque Fink ou DarwinPorts sont
install\'es. \\

\section{Pour la prochaine soutenance}

Le travail sur le noyau de Patchwork13! est \`a pr\'esent
th\'eoriquement termin\'e, et il est plus que s\^ur
que la version 0.1 {\it b\^eta} sera d\'eploy\'ee avant
la prochaine soutenance.
\\

\chapter{Interface graphique}

\section{Ce qui \`a \'et\'e fait}

L'interface graphique GTK+2 de Patchwork13! s'appelle pw13\_gtk et
est construite \`a l'aide de glade-2, une interface graphique de design
d'interface graphique. Cela permet de se concentrer sur l'ergonomie et
non sur les lignes de codes n\'ecessaire \`a la r\'ealisation d'une interface,
exp\'erience que nous avons d\'ej\`a faite auparavant et que nous ne
referons plus ! \\

En plus de cela, nous utilisons libglade, une librairie qui permet de
charger un fichier {\tt .glade} (le format de sauvegarde de glade,
qui est en fait du xml) {\bf dynamiquement} pour cr\'eer l'interface
graphique.

Cela \'evite la r\'ebarbative proc\'edure standard qui est de laisser glade
g\'enerer les sources du programme, ce qui oblige \`a recompiler le
programme \`a chaque modification, aussi minime soit-elle, de l'interface.

Avec libglade, il nous est arriv\'e de ne m\^eme pas avoir \`a fermer le
programme pour observer les modifications apport\'ees aux bo\^ites de
dialogues. Une fois modifi\'ee dans glade, il suffit de r\'e-ouvrir la
bo\^ite de dialogue pour voir les changements.

Cela est \'evidemment tr\`es pratique et permet de se concentrer sur les
fonctionnait\'es de l'interface et de ne pas perdre de temps sur sa mise
en oeuvre. \\

\newpage

\subsection{Bo\^ite \`a outils}

La bo\^ite \`a outils est la fen\^etre principale du programme,
elle permet d\'ej\`a de parcourir l'arborescence des patches
install\'es sur le syst\`eme.

Pour cela elle parcours r\'ecursivement le r\'epertoire des librairies
de patches et affiche l'arborescence.\\


On peut \'egalement gr\^ace au menu en haut de la fen\^etre :
\begin{itemize}
\item
  Cr\'eer un nouveau patchwork, ce qui ouvre une nouvelle fen\^etre
  de patchwork et ajoute son nom (unique) dans la liste des fen\^etres
  (dans le menu ``View'').

\item
  Ouvrir une bo\^ite de dialogue pour charger un patchwork, qui
  ouvre une fen\^etre de patchwork portant le nom du fichier.
  Le chargement n'est pour l'instant pas support\'e puisqu'on ne peut
  pas encore afficher les patches du patchwork (il faudra attendre
  la prochaine soutenance ;)

\item
  Un menu {\it Fen\^etre} permet de mettre en avant une des
  fen\^etres de patchwork ouvertes.

\item
  Changer les pr\'ef\'erences du cluster (changements que l'ont
  peut \'eventuellement annuler si on s'est tromp\'e).
  Les pr\'ef\'erences se r\'esument pour l'instant au chemin des
  ex\'ecutables du cluster, et l'on peut cliquer sur un bouton pour
  les s\'electionner {\it via} une classique bo\^ite de dialogue de
  s\'election de fichier.

\item
  Afficher une fen\^etre avec la liste des serveurs du cluster,
  que l'on peut rafraichir avec un bouton, cela \'evitant de saturer le
  r\'eseau inutilement. Cette fonction utilise les fonctions du cluster,
  la liaison entre l'interface et le cluster a donc d\'ej\`a commenc\'e.

\item
  Lancer un serveur sur la machine locale, le serveur lanc\'e
  correspond \`a l'ex\'ecutable choisi dans les pr\'ef\'erences du
  cluster.

\item
  Afficher une bo\^ite de dialogue ``\`A propos'', exercice simple
  mais utile.

\item
  Quitter le programme, avec un message de confirmation. Cela
  peut \'eviter de perdre un travail en cours par erreur.

\end{itemize}
\rule{0cm}{0cm} \\

\subsection{Fen\^etre patchwork}

Lorsqu'on cr\'ee un nouveau patchwork ou que l'on en charge un existant,
une fen\^etre de patchwork s'ouvre. On peut mettre en avant une
fen\^etre en cliquant sur l'\'el\'ement portant son nom dans le
menu {\it Fen\^etre} pour la mettre en avant.
De plus, quand on ferme la fen\^etre,
l'\'el\'ement u menu est lui aussi retir\'e. \\

La fen\^etre d'un patchwork est encore sommaire et pas tres interactive, mais 
l'idee principale est install\'e. Nous pouvons voir des maintenant apparaitre 
un block dans la partie superieur de la fenetre principale. Celle ci
contiendra l'ensemble des patchs d'un patchwork avec les
liaisons entre eux et leurs parametres interne. \\

Dans la partie inferieure de la fenetre, on peu voir 
un block divis\'e en plusieur parties. Ces parties sont reductibles \`a
souhait. Elles contiennent, de droite a gauche, une zone ou l'on
verra appara\^itre le 
suivi du patchwork: messages d'information, erreurs\ldots  \\

La deuxieme zone servira 
\`a lancer le patchwork et contr\^oler pas \`a pas l'\'evolution du
patchwork dans le temps.
Une trackbar permet de voir l'avanc\'ee du patchwork et de se d\'eplacer
dans son  \'evolution aussi bien au niveau du temps que des processus. \\

Le dernier block est encore inutilis\'e mais nous permettra \`a 
l'avenir de pouvoir afficher des proprietes supplementaires utiles \`a
l'utilisation du patchwork ou des informations diverses. \\

La barre de menu presente dans la partie sup\'erieure de la fen\^etre
contient des bouttons n\'ecessaires et essentiels \`a la bonne
utilisation du patchwork13. Mis a part les fonctions de base
(open, save, quit..), l'affichage de fenetres annexes utiles.


\newpage

\section{Pour la prochaine soutenance}

Nous avons pu nous familiariser avec les outils parmi les plus efficaces
pour cr\'eer une interface graphique en cr\'eant des fen\^etres, 
bo\^ites de dialogue, et en g\'erant les \'ev\`enements (ou signaux sous GTK+).
\\

Nous alons maintenant \^etre confront\'es \`a des probl\^emes plus ardus
comme cr\'eer nos propres widgets (classes d'objets) et utiliser des plugins
qui viennent d'\^etre cr\'ees pour GTK+, tels que cairo et graphviz. \\


\subsection{Affichage du graphe}

Nous pr\'evoyons d'utiliser la librairie {\bf graphviz} (qui est la
r\'ef\'erence pour afficher un graphe) afin d'afficher notre graphe
de patches.

Nous nous sommes d\'ej\`a un peu renseign\'es sur les plugin GTK
et cairo existant pour graphviz mais ceux-ci \'etant tr\`es
r\'ecents sont encore tr\`es mal document\'es.

Ceux-ci n'ont toutefois pas l'air tr\`es dynamiques, il nous faudra
surement \'ecrire toutes les fonctions n\'ecessaire \`a l'interaction
entre le graphe et l'utilisateur, via la souris et le clavier.
\\

\subsection{Widget patches}

Nous allons devoir cr\'eer un nouveau widget GTK+ pour les patches.
Il faut que l'on puisse y voir le nom et le type du patch ainsi que
ceux de ses entr\'ees et sorties. On devrait aussi pouvoir
masquer/afficher toutes ces informations en un clic pour ne pas
encombrer l'espace de travail. \\

Il faudra \'egalement permettre aux patches de pouvoir cr\'eer une
interface GTK+ et ainsi d'interagir avec l'utilisateur.
Il faudra pour cela que les patches ayant une interface graphique
exportent une fonction pour cr\'eer l'interface et y connecter des signaux.

\newpage

\chapter{Maintenance et installation}

La maintenance des r\'epertoires des sources est une t\^ache tr\`es
importante au bon d\'eveloppement du projet, bien au del\`a d'une
question de confort, la vitesse de d\'eveloppement du projet en
d\'epend directement. \\

Les facilit\'es d'installation sont pour l'instant r\'eduites au
classique mais efficace \\
\rule{1cm}{0cm}
{\tt make install} \\

Il est bien s\^ur pr\'evu d'avoir des programmes d'installation tr\`es
accessibles pour les diff\'erentes plateformes support\'ees
(Linux RPM, Debian DEB, Gentoo ebuild, MacOS X, Win32\ldots).
\\

\section{Ce qui a \'et\'e fait}

Tous les programmes ont leur propre r\'epertoire o\`u les sources
sont s\'epar\'es des fichiers n\'ecessaires \`a {\tt autoconf} et
{\tt automake}.\\

Ces deux programmes se sont impos\'es par leur grande notori\'et\'e
et la facilit\'e de portage qu'ils offrent. Ils sont m\^eme disponibles
pour Widows avec MinGW/MSYS ! \\

Nous utilisons \'egalement {\tt libtool} pour compiler, lier et
charger dynamiquement les librairies de mani\`ere portable. \\

Ces trois programmes sont actuellement utilis\'es pour
compiler chaque programme du projet et sont d'une tr\`es grande
aide car nous n'avons eu presque aucun probl\`eme pour compiler sur
diverses plateformes une fois la configuration bien faite. \\


\newpage

\subsection{Configuration}

L'installation et la maintenance des programmes se fait gr\^ace \`a 
autoconf et automake.

Ces deux programmes permettent de g\'en\'erer a partir de peu d'instructions
des Makefile complexes qui s'adaptent \`a toutes les plateformes UNIX,
et m\^eme \`a Windows une fois dot\'e de MinGW et MSYS.\\

La proc\'edure \`a suivre pour l'installation est la proc\'edure classique
d'installation d'un package sous UNIX : \\
{\tt

  ./configure

  make

  make install

}
\rule{0cm}{0cm} \\

Ainsi le noyau a \'et\'e compil\'e avec succ\`es sur les
plateformes suivantes :
\begin{itemize}
\item{Gentoo Linux sur x86}
\item{Gentoo Linux sur PowerPC}
\item{MacOS X 10.4.2 (Darwin) sur PowerPC}
\item{NetBSD sur x86}
\item{Debian 3.0 sur x86}
\item{Ubuntu (Breezy Badger) Linux sur x86}
\item{Windows XP avec MinGW/MSYS sur x86}
\end{itemize}
\rule{0cm}{1mm} \\

La librairie standard ainsi que l'interface graphique ont \'egalement
\'et\'e compil\'ees avec succ\`es sur chacune de ces plateformes, \`a
l'exception de Windows (MinGW/MSYS), qui n'est toutefois pas loin de
fonctionner.

Meme si elle repr\'esente un march\'e ph\'enomenal, 
la plateforme n'a pas \'et\'e notre priorit\'e pour cette soutenance
car il n'y a pas beaucoup de travail \`a fournir pour que
la plateforme soit support\'ee
mais nous concentrer pour l'instant sur UNIX nous a apport\'e
une portabilit\'e tr\`es largement sup\'erieure. \\

D'autre part il est beaucoup plus facile de porter un projet UNIX
sous windows que de faire l'inverse, qui est m\^eme tout \`a fait
inenvisageable. \\


\newpage

\subsection{CVS}

La maintenance et le travail en \'equipe ont \'et\'e
consid\'erablement simplifi\'es par
l'utilisation de CVS (Concurrent Version System),
qui permet d'enregistrer toutes les versions
successives des fichiers du projet.

Cela permet aussi de travailler beaucoup plus simplement \`a plusieurs
sur un m\^eme projet, puisqu'on peut m\^eme (bien que cela ne soit pas
recommand\'e) travailler \`a plusieurs sur un m\^eme fichier, CVS se
chargeant de {\it m\'elanger } les deux et d'indiquer les \'eventuels
conflits.\\

CVS donne \'egalement une sensation de s\'ecurit\'e tr\`es encourageante
car on ne risque pas de perdre un fichier ou de modifier irr\'emediablement
un fichier puisque on peut \`a tout moment r\'ecup\'erer toutes les versions
 pr\'ec\'edentes. \\

La base CVS que nous utilisons est gracieusement
(et gratuitement) h\'eberg\'ee par SourceForge.net
que nous tenons donc \`a remercier tr\`es chaleureusement ! \\


\section{Pour la prochaine soutenance}

\subsection {Porter les programmes pour Win32}
Nous pr\'evoyons de pouvoir compiler et lancer tous les programmes
sous Win32 d'ici la prochaine soutenance.

Le noyau est compl\`etement compatible avec Win32 et produit une DLL
utilisable par d'autres programmes. \\

C'est ce que nous devrons obtenir pour l'ensemble des patches de la
librairie standard, o\`u chaque patch devra \^etre une DLL.
C'est une op\'eration qui est loin d'\^etre \'evidente vu les contraintes
impos\'ees pour compiler une DLL sous Wind0ws. \\

Il reste actuellement des probl\^emes qu'il faudra r\'esoudre
pour pouvoir compiler l'interface graphique avec GTK+. \\

Il y aura certainement de petits changements a apporter aux
fonctions r\'eseau du cluster, ainsi qu'\`a la mani\`ere de compiler
puisque nous ne l'avons jamai compil\'e sous Win32. \\

\chapter{SDL}
\section{introduction a sdl}

La librairie SDL a \'et\'e cr\'e\'ee afin de faciliter la programmation
pour les jeux-videos.
Nous allons l'utiliser ici afin de gerer les entr\'ees et sorties audio.
 Pour l'instant la lecture d'un son via le patchwork n'est pas complete,
 mais on sais d\'ej\`a comment utiliser cette librairie a bon escient.
\\
Nous savons \`a pr\'esent que le patch pour le son sera
inclus au patchwork13 tres bientot.
\\
\\
L'utilisation de la librairie SDL nous permettra par la suite d'afficher
une fen\^etre OpenGL
et l'utiliser dans un patch.
On pourrai meme envisager d'utiliser SDL a des fin beaucoup plus optimiste.
\\
SDL est une librairie multi-OS et multi taches.
Elle nous permettra donc, comme GTK+, de 
porter le patchwork sur tous les systemes d'exploitation. ``Vive SDL''



\chapter{Cluster}

\section{Ce qui a \'et\'e fait}

\subsection{Pr\'esentation g\'en\'erale et d\'efinition du cluster}

Un cluster est un groupe de serveurs ind\'ependants fonctionnant comme un seul
et m\^eme syst\`eme.

Un client dialogue avec un cluster comme s'il s'agissait
d'une machine unique. Ce syst\^eme permet donc d'augmenter consid\'erablement
les performances lors du traitement d\'esir\'e. \\

\subsection{Fonctionnement g\'en\'eral}

Le cluster est divis\'e en deux parties :

Un client, lanc\'e sur la machine sur laquelle on veut utiliser patchwork13, et
 un nombre ind\'efini de serveurs qui sont l\`a pour faire les t\^aches de 
traitement demand\'ees par le client. \\

La communication entre le client et les servers s'effectue en deux temps :

\begin{itemize}
\item une identification en UDP
\item Connection et \'emission en TCP
\end{itemize}
\rule{0cm}{0cm} \\

\subsubsection{Identification}

Le client poss\`ede une procedure effectuant un envoi de paquet sur 
l'adresse de broadcast. Cette action est r\'ealisable depuis le bouton
refresh de l'interface graphique.

 Les serveurs en \'ecoute re\c{c}oivent
donc le paquet \'emit par le client et renvoient un paquet, contenant
entre autre le score du serveur, au client pour signaler leur
pr\'esence.

 Une fois le paquet  d'identification re\c{c}u,
le client range le serveur dans une liste
de servers  en fonction du score qu'il a transmis.
D\`es lors l'identification est 
termin\'ee pour les serveurs connect\'es.
\\[1cm]

\subsubsection{Connection et \'emission en TCP}

Lors d'un refresh, le client parcourt la liste des serveurs \`a la recherche de
serveurs non connect\'es en TCP.

D\`es qu'un tel serveur est rep\'er\'e, le 
client initialise avec lui une connection en TCP. A ce moment l\`a, le client 
et le serveur sont compl\`etement op\'erationnels pour \'echanger toutes
sortes de messages. \\


\subsection{D\'etails des fonctions importantes}

\subsubsection{Calcul du Score}

Comme pr\'ecis\'e plus haut, le serveur envoie son score au client. Cette 
fonction permet au client de classer les serveurs en fonction de leurs 
performances.

Ce score est obtenu par la multiplication du pourcentage d'utilisation du CPU 
par sa puissance et de m\^eme pour la Ram, en plus clair :
\\
``score\_de\_la\_machine = (\%charge\_CPU * CPU) * (\%charge\_RAM * RAM)''
\\
Le plus dur reste \'a faire, soit r\'ecup\'erer les donn\'ees composantes de 
l'\'equation. Cela varie en fonction des syst\`emes, donc nous avons d\'ecid\'e
de ne faire que sous Unix pour l'instant, les valeurs de CPU et de RAM se 
trouvent dans les fichiers ``/proc/cpuinfo'' et ``/proc/meminfo''.

Pour extraire les informations j'ai utilis\'e la methode avec
un fscanf en utilisant les 
options d'ignorances ``\%*s'' me permettant d'arriver \'a la valeur voulue
qui est
toujours au m\^eme endroit. Pour le pourcentage de la charge de la RAM, elle 
est \'ecrite dans un fichier, donc m\^eme m\'ethode que juste au dessus.

 La partie d\'elicate est le calcul de la charge du cpu, il faut utiliser un 
fichier ``/proc/stat/'' qui donne des statistiques sur le nombre de Tick en 
comparant avec un petit intervalle pour pas trop ralentir le processus de 
Broadcast. Soit on r\'ecupere les donn\'ees n\'ecessaire plusieurs avec la
 m\^eme m\'ethode qu'au dessus encore puis on renvoie en faisant une moyenne.
\\


\subsubsection{Protocole et extraction des messages}

Pour que les messages envoy\'es soient compr\'ehensibles, il est necessaire de 
poss\'eder un protocole.
Il \`a \'et\'e d\'efini de la mani\'ere suivante :

un ``\#'' pour annoncer le debut du message, un chiffre repr\'esentant
la classe
du message (message syst\^eme, message relatif \`a un patch ...), un ``\#'' 
s\'eparateur, un autre chiffre servant \`a d\'efinir le type de message
transmi
(par exemple : \#1\#1\# repr\'esente le paquet de broadcast), un ``\#'' et
pour finir la cha\^ine du type d\'efini par les chiffres pr\'ecedents.
Il para\^it d\`es lors n\'ecessaire de poss\'eder une fonction pour
extraire les 
param\`etres. On a donc cr\'ee la fonction {\tt n\_ieme} :

Fonction simple mais tr\`es importante dans le traitement des paquets 
re\c{c}us. Elle permet de retourner une cha\^ine entre deux occurences \# d'une
chaine d'un paquet ce qui a pour avantage de traiter facilement  les 
donn\'ees ou message re\c{c}u. Une double boucle l'une \`a la 
suite de l'autre permet de rep\'erer les \# et de construire la cha\^ine 
r\'esultante.
\\

\subsubsection{Fonction de Broadcast}
Le fonctionnement global de cette fonction est d'envoyer un paquet sur le 
broadcast, de recevoir des r\'eponses en UDP de la part des serveurs 
re\c cevant le paquet de broadcast et d'ajouter ces serveurs \'a la liste des 
serveurs. De leurs c\^ot\'e les serveurs poss\`edent de m\^eme une liste de 
clients. \\

C'est une liste cha\^in\'ee, elle  a \'et\'e implement\'e pour permettre de 
classer les diff\`erents serveurs ou clients(en fonction du poste o\`u l'on se 
trouve). Elle est \`a peu de chose pr\`es \'equivalente sur l'un et l'autre. 
Elle contient cot\'e client: \\

\begin{itemize} 
\item l'adresse ip d'un serveur 
\item le score 
du serveur 
\item le descripteur du socket 
\item un bool\'een pour savoir si la connection tcp est activ\'e ou pas 
\item un pointeur vers l'\'el\'ement suivant 
\end{itemize} 
\rule{0cm}{0cm} \\

Du c\^ot\'e serveur, on a l'\'equivalent sans le score.
 
Pour g\'erer cette liste plusieurs fonctions sont pr\'esentes pour ajouter, 
supprimer, trier, v\'erifier si un \'el\'ement est pr\'esent, afficher la 
liste, \ldots 

Toutes ces fonctions sont classiques et reprennent les algorithmes vus en 
algorithmique sur les listes cha\^in\'ees. Seul l'ajout est peut-\^etre un peu 
bizarre car il s'effectue en fin de liste pour optimiser sans passer par la 
fonction de v\'erification de pr\'esence d'\'el\'ement soit autant ensuite 
ins\'erer \`a la fin du parcours. \\

\subsubsection{Connection en TCP}
Pour une gestion ais\'e, quatres fonctions ont \'et\'e cr\'ees :
TCPsend, closeTCPsocket, TCPreceive et TCPinit. 
C\^ot\'e serveur, ce dernier est en listen(mode TCPinit) et attend donc une 
connection, pour ensuite ce mettre en mode TCPreceive.

C\^ot\'e client on commence par initialiser des connections vers les serveurs 
inscrits dans la liste en \'evitant ceux qui sont d\'ej\`a connect\'es.
Ensuite le client se met en r\'eception \`a son tour.

D\`es lors les connections entre le client et tous les serveurs connect\'es 
sont op\'erationnelles pour envoyer et recevoir des donn\'ees. \\

\newpage

\section{Ce qui est \`a faire pour la prochaine soutenance}

La base du cluster est pr\'esentement op\'erationnelle. Il faut maintenant 
rendre cette base utilisable pour patchwork13.
Pour ce faire nous aurons besoin de :

\begin{itemize}
\item pouvoir envoyer tout type de donn\'ee et pas seulement des cha\^ines 
\item faire un serveur \'etant capable d'avoir un contr\^ole total sur les 
patchs : cr\'eation, destruction, connecter les sorties de ses patchs sur 
d'autres patchs pouvant \'etre soit en local soit sur un autre noeud du 
cluster, donc par le r\'eseau.
\item g\'erer le fait que le patch demand\'e n'\'existe pas sur le server 
selectionn\'e et donc choisir un autre server.
\end{itemize}
\rule{0cm}{1cm} \\


\chapter{Site Web}

Le site web est d\'ej\`a bien lanc\'e puisque chercher ``patchwork13''
sous Google et faire ``J'ai de la chance'' envoie directement sur
notre page d'accueil !

Cela permet notemment de taper seulement ``patchwork13''
directement dans la barre d'adresse du merveilleux Firefox pour
arriver sur notre site. \\

Le site est h\'eberg\'e par SourceForge.net [http://patchwork13.sf.net/]

et dispose \'egalement d'un miroir chez lowh.net [http://patchwork.lowh.net]

que nous tenons donc \`a remercier pour ce service gratuit. \\

Le site permet de naviguer sur quelques pages grace \`a un menu lat\'eral
situ\'e \`a gauche.

L'ensemble des pages du site sont en fait g\'en\'er\'ees par un
seul script PHP qui se charge de transmettre la page demand\'ee et
d'accorder le menu en d\'esactivant le lien de la page en cours et
en mettant le titre de cette page en haut, juste sous le logo de
Patchwork13!. \\

Le site comporte actuellement les pages suivantes :
\begin{itemize}
\item {\bf About}

  Une page d'introduction, d\'ecrivant sommairement le projet
  pour les gens qui ne le connaissent pas.
  C'est la page par d\'efaut du site. \\

\item {\bf News}

  Une page affichant les derni\`eres informations et nouvelles
  \'evolutions du projet. Un formulaire et un script pour
  administrer les news est en cours d'\'elaboration. \\

\item {\bf Forum}

  Un lien vers les forums du projet, fournis par SourceForge. \\

\item {\bf Doc}

  Une page liant vers les diff\'erentes documentations du projet,
  cela comprend le cahier des charges, ainsi que les documentations
  de chaque package g\'en\'er\'ees gr\^ace \`a {\bf Doxygen}. \\

\item {\bf Download}

  Une page expliquant comment se procurer les sources.
  
  Pour l'instant aucun paquet n'est pr\^et et la seule mani\`ere
  d'obtenir Patchwork13! est d'en r\'ecup\'erer les sources via CVS.

  La page donne les commandes \`a taper, pour une connection
  anonyme et pour un compte sourceforge. \\

\item {\bf Authors}

  La liste des gens ayant contribu\'e au projet. \\


\end{itemize}

\chapter{Conclusion}

D\`es la premi\`ere soutenance, le projet est en avance sur son calendrier
et est d\'ej\`a dot\'e d'un r\'eseau de base capable de g\'erer plusieurs 
clients et plusieurs servers avec une interface qui se lie d\'ej\`a
au noyau (pw13) et au cluster.

Du c\^ot\'e de l'interface, cette derni\`ere propose un environnement
assez fourni (boites de dialogue, pr\'ef\'erences \ldots) 
permettant d'ores et d\'ej\`a un minimum d'interaction. \\

Le travail que nous avons fourni jusque l\`a constitue une base solide
pour la suite de l'avancement du projet.

D'autre part on aurait pu s'attendre \`a voir quelques probl\`emes
de coop\'eration puisque toutes les objectifs que nous nous \'etions
fix\'es pour cette soutenance \'etaient \`a r\'ealiser \`a deux,
or il n'en est rien. \\

Le groupe a fait preuve d'une tr\`es bonne coh\'esion et les
r\'eunions de groupe ont \'et\'e multiples et ont \`a chaque fois
b\'en\'efici\'e du charme magique du d\'ebugage de minuit, non
pas que nous \'etions en retard mais plutot parce que nous
aimons vraiement cette \'etrange et f\'ebrile insomnie. \\

Pauvres de nous !

\end{document}
