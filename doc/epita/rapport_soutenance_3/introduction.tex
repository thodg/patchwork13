\chapter{Introduction}

Voici venue la troisi\`eme soutenance de {\bf Patchwork13!},
notre puissant outil
de synth\`ese modulaire. Rappellons bri\'evement le but du projet :
permettre la g\'en\'eration et la manipulation de donn\'ees en
temps r\'eel par la connection de {\em plug-ins} que nous appellons des
{\bf patches}, cela tr\`es simplement et \`a la souris.\\

Nous d\'eveloppons \'egalement un syst\`eme de clustering pour r\'epartir
les patches sur diff\'erentes machines afin de multiplier les performances,
cela sans m\^eme que l'utilisateur aie \`a configurer quoi que ce soit \`a
part lancer le {\em daemon} sur les machines qu'il souhaite utiliser.
Le but est qu'en tirant des fleches \`a la souris, l'utilisateur controle
de mani\`ere transparente tout les serveurs.\\
\\

\section{Planning}
Notre planing a \'et\'e largement respect\'e tant au niveau du cluster que 
de l'interface, des patches SDL et l'OpenGL et des patchs son. Cela a
\'et\'e facilit\'e par l'avance que nous avions prise par exemple sur les
patches OpenGL, mais aussi compliqu\'e par les nouvelle fonctionnalit\'es
que nous avons rajout\'ees en plus de celles d\'ecrites dans notre cahier des
charges. Cela comprend l'exportation et l'importation dans des fichiers XML
et l'avanc\'ee majeure que repr\'esente la possibilit\'e d'associer aux
patches des {\em m\'ethodes} qui leurs sont propres.\\

De plus, l'enregistrement et du chargement des patchworks 
permet de ne plus perdre de temps lorsqu'on veut reprendre
un patchwork et/ou le modifier, apportant un grand confort
d'utilisation, surtout pour les tests ce qui acc\'el\`ere
fortement le d\'eveloppement.\\

Concernant le cluster, l'impl\'ementation des messages r\'eseau que nous 
avions pr\'evus est compl\`ete, ce qui permet \`a Patchwork13! d'\^etre
compl\`etement utilisable \`a travers un r\'eseau TCP/IP.\\

\newpage

Pour la SDL, nous avons maintenant une sortie sons qui fonctionne parfaitement 
et nous permet, coupl\'e avec les patchs sons cr\'ees, de pouvoir d\'ej\`a 
produire un rendu sonore tr\`es satisfaisant.\\
\\

\section{Exp\'erience}

Entre ces deux soutenances, beaucoup de travail a donc \'et\'e abattu.
Certains ont d\'ecouvert le confort des salles machines du sous-sol
Pasteur, dont la temp\'erature se pr\^ete tout \`a fait \`a un court
mais r\'eparateur sommeil lorsque les joies de la programmation
c\'edaient face aux charmes de Morph\'ee.\\

Le ryhtme de travail est donc vraiement bien install\'e et il n'y a aucune
raison pour que cela ne continue pas \`a s'am\'eliorer. 
