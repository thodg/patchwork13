\chapter{Interface graphique}

\section{Le patchwork}
\subsection{Nouvelles fenetres}
\subsubsection{Sauvegarder \`a la fermeture}
Petite fen\^etre suppl\'ementaire pour les t\^etes en l'air.
\`A la fermeture d'un patchwork, il y a maintenant une fen\^etre
de validation demandant
si l'on ferme, on sauve ou on annule. Simple mais efficace.
\\
\\
On remerciera la magie de l'informatique. Merci.
\\
\\

\section{Les patchs}
\subsection{Suppression}
\par
Mieux que l'ajout de patch, visible a la soutenance precedente, la
suppression des patchs.
\\
\\
Et oui, nous pouvions ajouter des patchs dans le patchwork grace a un
glisser-deposer lors de la derniere soutenance. Nous pouvons maintenant
les supprimer grace a un clique de la molette sur le titre du patch.
\\
\\
Grace a une technologie avanc\'ee, nous avons meme la possibilitee
d'enlever un patch pendant qu'un patchwork est lanc\'e. Il a fallu
modifier quelques \'el\'ements de la librairie noyau, qui est du coup
beaucoup plus robustes.
\\
\\

\section{Les connections}
\par
Mieux que les connections, les deconnections..
\\
\\
En effet, il est interessant de pouvoir connecter deux patchs mais il est
aussi tres important pour l'utilisateur de pouvoir enlever des connections.
Ca y est ! il a le choix de soit deconnecter sans trace, c'est a dire sans
garder la fleche, en droppant l'entree, soit de garder une trace, fleche
gris\'ee, juste en cliquant sur l'entree ou la sortie. La reconnection se
refait simplement en tirant le lien.
\\
\\
\par
A l'avenir, il sera possible de desactiver une connection et de la reconnecter
en un clic. Une amelioration pour la version finale du Patchwork13.
\\
\\

