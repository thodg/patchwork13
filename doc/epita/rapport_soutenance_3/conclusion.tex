\chapter{Conclusion}

\section{Les accomplissements}
Patchwork13! \`a la troisi\`eme soutenance nous procure beaucoup de 
satisfactions \'etant donn\'e qu'il est d\`es lors utilisable que nous nous
sommes amus\'es avec pendant d\'ej\'a plusieurs heures \`a faire des sons
vraiement \'etranges, et \`a les modifier en {\em live}.
Nous en avons d'ailleurs d\'ej\'a r\'ealis\'e un au BDE qui
\`a \'et\'e assez appr\'eci\'e, cela pendant plus de vingt minutes
sans interruption, ce qui prouve la stabilit\'e du projet.\\
\\
Apr\'es deux semaines de pseudo vacances pass\'ees \'a d\'evelopper
notre projet, nous avons beaucoup avanc\'e vers ce que nous 
ttendons de patchwork13!. Nous pouvons cr\'eer des patchwork,
y d\'eposer des patchs alors que l'on joue, les relier entre eux, 
les suprimer, sauvegarder des patchwork, les charger, mixer du son
et de la 3d en live localement ou \'a traver le r\'eseau...
en somme beaucoup de choses !\\
\\

\section{Et le futur..!}

Il ne nous reste plus qu'\`a terminer l'interface graphique
en y int\'egrant le cluster, clarifier le code de certaines
fonctions du r\'eseau et \'ecrire un grand nombre de patchs
pour \'eventuellement s\'eduire des artistes en herbe et d'autres
programmeurs pour qu'ils contribuent au projet en \'ecrivant des patches !\\

Le portage ne devrait pas poser de probl\`eme puisque nous d\'eveloppons
tr\`es r\'eguli\`erement sur (liste non exhaustive) sur architecture PowerPC,
sous Gentoo et Mac OS X / Darwin, et sur architecture x86 sous Gentoo,
Debian, NetBSD, OpenBSD. Le portage windows \'etait fonctionnel il y
moins d'un mois et ne devrait pas \^etre trop compliqu\'e puisque nous
n'avons pas rajout\'e beaucoup de librairies (libxml2).\\

\newpage

Un programme d'installation pour Windows est presque pr\^et. Pour UNIX
nous fournissons le format classique des packages autoconf/automake
({\tt ./configure; make; sudo make install}) qui est compatible avec
tous les *NIX que nous avons pu tester.\\

L'ambiance du groupe est toujours au beau fixe nous permettant d'\^etre 
confiants sur le travail \`a effectuer pour la soutenance finale.
Le rythme de travail n'a cess\'e de s'acc\'el\'erer depuis le
d\'ebut de l'ann\'ee et ne devrait pas \'echapper \`a cette loi
pour ces deux derniers mois. D\'efinitivement prometteur.

