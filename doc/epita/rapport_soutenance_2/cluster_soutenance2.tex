\chapter{Cluster}

\section{Pr\'esentation g\'en\'erale et d\'efinition du cluster}

Un cluster est un groupe de serveurs ind\'ependants fonctionnant comme un seul
et m\^eme syst\`eme.

Un client dialogue avec un cluster comme s'il s'agissait
d'une machine unique. Ce syst\^eme permet donc d'augmenter consid\'erablement
les performances lors du traitement d\'esir\'e. \\

\section{Rappel du fonctionnement g\'en\'eral}

Le cluster est divis\'e en deux parties :

Un client, lanc\'e sur la machine sur laquelle on veut utiliser patchwork13, et
 un nombre ind\'efini de serveurs qui sont l\`a pour faire les t\^aches de 
traitement demand\'ees par le client. \\

La communication entre le client et les servers s'effectue en deux temps :

\begin{itemize}
\item une identification en UDP (broadcast)
\item Connection et dialogue entre client et server en TCP 
\end{itemize}

\newpage

\section{Ce qui a \'et\'e fait}

\subsection{D\'efinition des types utilis\'es}
Pour avoir un cluster simple mais complet, nous avons d\'efinis diff\'erents
types qui sont en fait des structs :\\


\begin{itemize}
\item
   Le type server contient un descripteur de la socket qui est en \'ecoute
permannente de nouvelle connection TCP, ainsi qu'une liste de structures 
client.\\

\item
   Le type client (client du server) contient un descripteur de socket sur 
laquelle le client (l\`a o\`u est lanc\'e patchwork13, o\`u l'output d'un 
patch) est connect\'e, un pointeur sur le patchwork qui tourne sur le server, 
un pointeur sur le thread correspondant aux communications avec le client 
connect\'e., une liste d'outputs (outputs reseau), une liste d'inputs (inputs
reseau) et un pointeur sur le server auquel il appartient.\\

\item
  Le type input r\'eseau contient un pointeur sur input, un descripteur de 
socket pour savoir avec qui cette input est connect\'ee,un pointeur sur le 
thread utilis\'e par cette output et un pointeur sur le client correspondant.\\

\item 
   Le type output r\'eseau contient un pointeur sur output, un descripteur de 
socket pour savoir avec qui cette output est connect\'ee,un pointeur sur le 
thread utilis\'e par cette output et un pointeur sur le client correspondant.\\
\end{itemize}
  

\subsection{Fonctions "communes"}
Le client et le server ont besoin de quelques fonctions en plus que celle qui 
ont d\'ej\`a� \'et\'e cr\'ees pour la premiere soutenance :\\

\begin{itemize}
\item
   Fonction waitsock\_data : cette fonction bloquante attends que que le buffer 
soit rempli avec un certain nombre d'octets gr\^ace \`a� la fonction syst\^eme 
ioctl. Cette fonction est utilis\'ee dans la majorit\'ee des fonctions que nous 
avons pr\'esent\'e pr\'ec\'edemment.\\

\item
   Fonction string\_send : \'etant donn\'e qu'on ne peux pas envoyer 
directement des cha\^ines par le reseau, on a d\^u cr\'eer cette fonction.
 Cette fonction prend un char* et un descripteur de socket en param\`etres,
 envoie un 
premier entier qui repr\'esente la longueur de la cha\^ine, puis la cha\^ine 
elle m\^eme.\\

\item
   Fonction string\_reception : \'etant donn\'e qu'on ne peux pas recevoir 
directement des chaines par le reseau, il a fallu impl\'ementer une fonction 
r\'ealisant cela. Cette fonction prend un descripteur de socket en 
param\`etre, attend un premier entier qui repr\'esente la longueur de la 
chai\^ine, puis apelle la fonction waitsock\_data en lui passant la
 longueur de 
la chaine en param\`etre. Une fois la chaine charg\'ee dans le buffer, on 
apelle la fonction recv pour recevoir notre cha\^ine
 que l'on retourne ensuite.\\
\end{itemize}

\newpage

\subsection{Traitement des messages r\'eseau}

Le client est le "ma\^itre" du cluster et, a donc besoin de pouvoir ordonner 
aux server d'effectuer les actions que le client aura command\'e dans 
patchwork13. Pour se faire, le server poss\`ede un switch pour traiter les 
messages qu'il re\c{c}oit. \\
\\
On a du cr\'e\'er une fonction waitsock\_data qui est une fonction bloquante
qui attend que le buffer soit rempli d'un certain nombre d'octets. Dans le
{\tt switch} cete fonction est utilis\'ee pour recup\'erer un premier entier,
correspondant a l'identifiant du message qui va suivre.
En fonction du {\tt case} dans lequel
l'entier entre, il demande \`a une des fonction de s'\'ex\'ecuter. 
Les fonctions qui ont \'et\'e realis\'ee sont :\\

\begin{itemize}
\item 
  create\_patch\_client\_of\_server, qui demande
  \`a un server de creer un patch
  en lui envoyant la classe du patch ainsi que son nom dans la classe. Une fois
  fait, le server renvoie au client l'adresse du patch qu'il a cr\'e\'e, ceci 
  afin que la liste des patchs par server, maintenue par le client, soit \'a 
  jour.\\
  
\item
  destroy\_patch\_client\_of\_server, qui detruit un certain patch et efface 
  toutes les input/output connect\'es correspondants \`a ce patch dans les 
  listes d'outputs/inputs du client du server correspondant.\\
  
\item
  int connect\_patch\_local\_client\_of\_server, qui connecte localement une 
  output et une input appartenant \`a des patchs locaux.\\
  
\item
  connect\_patch\_distant\_client\_of\_server, qui demande \`a un server de 
  connecter une des output d'un de ses patch \`a une input d'un patch
  lanc\'e sur un autre server.
  Pour se faire, une connection TCP est etablie entre les deux
  server avec un thread de chaque c\^ot\'e pour g\'erer
  les communications entre ces inputs/outputs.
  \`A travers cette connection, le premier client envoie au
  second une demande de bind\_patch\_input\_client\_of\_server ce qui
  a pour effet de tenir a jour les listes d'input pour le second server.
  Ensuite, le premier
  client met \`a jour sa liste d'outputs.\\

\item
   ask\_patch\_start\_client\_of\_server, sert \`a demarrer
   un patch \`a un certain
   temps donn\'e.\\

\item 
   ask\_patch\_stop\_client\_of\_server, sert \`a stopper
   l'\'ex\'ecution d'un 
   patch.\\
\end{itemize}


\subsection{Messages du Client}


Au niveau du Client(le pc principal sur le reseau celui
o\^u est l'utilisateur) il faut envoyer les messages
permettant de r\'ealiser les actions.
Pour discuter avec les Serveurs, le Client utilise
des messages pr\'ed\'efinis que l'on
envoit au different
protagoniste en jeux. Pour cela on utilise les messages
pr\'ef\'edinis dit pr\'ec\'edement
dans la r\'eception de message donc : <<cr\'e\'e patch>>,
<<d\'etruire patch>>, \ldots
A partir de \c{c}a, chaque fonction a son message
et on prend en param\^etre ce que
le message \`a besoin. Puis on envoit le tout,
il y a 2 fonctions aidant \`a cela, une
envoyant un entier l'autre des chaines de charact\`eres,
et puis des ``send'' normal
pour les structures classiques.\\


\section{Ce qu'il reste \`a faire}
Nous avons quasiement fini le cluster et sommes donc en avances sur les 
pr\'evisions. Cependant il reste une fonction majeure a coder : 
{\tt ask\_pump\_patch}, qui aura pour r\^ole de faire la
remont\'ee recursive aux
p\`eres de la demande de pump. Pump sur un patch a pour effet de le faire 
calculer ce dernier en pompant ses inputs. \\

A cette soutenance, niveau cluster, il est possible de lancer diff\'erents 
patchs sur diff\'erents servers et d'actionner ses patchs. Par exemple, il est 
possible de lancer un patch son sur un ordinateur, un patch opengl sur un autre
et de faire fonctionner les deux patchs en m\^eme temps.\\

Cette fonction  nous permettra donc de clore le cluster et de permettre un 
fonctionnement total entre les patchs repartis sur differents server.\\
