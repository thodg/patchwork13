
\chapter{Librairie de patches standards}

\section{Donn\'ees}

Tous les types de la librairie standard ont \'et\'e chang\'es pour
s'adapter \`a la nouvelle gestion des donn\'ees, qui permet d\'esormais
de g\'erer la taille des donn\'ees et les tableaux
(voir \ref{noyau_donnees} {\bf Gestion des donn\'ees},
 page \pageref{noyau_donnees}).\\


\section{Type vecteur}

\section{Exemple de nouveau type : Vect4f}

\`A la base nous avons cr\'e\'e ce type pour l'OpenGL, afin de
contenir les coordonn\'ees d'un vecteur pour la 3D.
Gr\^ace au type de base d\'ej\`a existant et en les
adaptant on peut facilement en inventer.\\

Par exemple pour vect4f on cr\'e\'e un dossier avec un
petit header dedans contenant des {\tt \#define} pour
d\'efinir les nouveaux input et output mais aussi
sp\'ecialement ici une fonction lui permettant de faire
une variable facilement tout en la remplissant.
Bien s\^ur ensuite une multitude de patches accompagne
ce type pour lui faire faire les fonctions courrantes
telles que addition, soustraction, constante, affichage.\\

\subsection{le type}
Pour ce type on voulait avoir un tableau statique de 4 cases
m\'emoires. Pour que se fasse, on utilise une fonction se trouvant
dans le noyau du projet qui initialise un tableau (de 4 cases, logique)
et on le remplis avec des \'el\'ement g\'en\'eriques pris en
param\^etres que l'on caste avec l'union contenant tous les
types tr\`es simples. \\

\subsection{les patchs d\'ependants}
Les patchs sont g\'er\'es comme les autres avec la fonction
``pump'' et ``init''.
Le type \'etant un peu plus complexe pour cet exemple,
il a fallu ajouter des ``defines'' au types pour ais\'ement
manipuler les donn\'ees de chaque case m\'emoire
dans les inputs et outputs.\\


\section{Extensions possibles}
Il y a une infinit\'e de patches \`a \'ecrire, surtout que l'on peut
utiliser des types plus complexes. L'objet de notre projet est pourtant
plus leur mise en route et leur moyens de fonctionnement car nous
pensons que cela encouragera plus de gens \`a utiliser
la librairie et donc \`a \'ecrire d'autres patches.\\

Nous \'ecrirons donc surement quelques patches pour illustrer le
fonctionnement du noyau, du cluster, de l'interface graphique et des
librairies que nous \'ecrivons (SDL OpenGL et son), mais la librairie
standard n'aura plus de travail officiellement pr\'evu pour cete ann\'ee.\\
