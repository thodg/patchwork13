\chapter{Conclusion}

Le projet Patchwork13 se dessine de plus en plus vers le r\'esultat 
recherch\'e et nous sommes en avance sur le planning ce qui nous rend confiants
sur la finalisation future du projet. 

Au niveau de l'interface, il est d\'esormais possible de cr\'eer des patchs 
dans le patchwork \`a la souris et de les relier entre eux \`a la souris, ceci 
coupl\'e avec un effet visuel de fl\'eches repr\'esentant les liaisons 
cr\'e\'es.

Du c\^ot\'e cluster, on peut faire cr\'eer des patchs \`a un server, les 
d\'emarrer, les arr\^eter et les d\'etruire. En resum\'e un client peut par
exemple demander \`a un server de d\`emarrer un patch son et donc par exemple
de pouvoir faire un concert en salle machine o\`u chaque machine jouera un 
instrument diff\'erent, tout cela command\'e par le client. 

Tout comme a la premi\`ere soutenance, le groupe a fait preuve d'une bonne 
cohesion ce qui nous a permi de coder plut\^ot efficacement (oui plut\^ot car
 des fois on avait du mal a cause a cause de ... enfin tout le monde s'en 
doute ... *sifflotte*). 

Il nous reste donc \`a finaliser le cluster pour que l'interop\'erabilit\'e 
entre les patchs soit complete et de lier le cluster à l'interface afin 
d'obtenir le projet compl\'etement utilisable. Suivront de nombreux patchs 
pour bien illustrer les possibilit\'es du projet.  

Sur ce, rendez-vous \`a la prochaine soutenance !
 
