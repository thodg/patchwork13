\chapter{Interface graphique GTK+2}

L'interface graphique a beaucoup progress\'e depuis la derni\`ere soutenance.
Il est d\'esormais possible de cr\'eer des patches, de les d\'eplacer
et m\^eme de les connecter entre eux, le tout \`a la souris.\\

\section{Glisser - d\'eposer}

Toutes ces op\'erations peuvent en fait se rapporter \`a des
glisser - d\'eposer ({\em drag and drop} en anglais). Nous avons donc
fait grand usage de la m\'ethode fournie par GTK pour effectuer des
glisser - d\'eposer. Malheureusement cette m\'ethode est tr\`es tr\`es
(tr\`es) mal document\'ee (contrairement au reste de GTK) et il a fallu
avancer \`a t\^atons, d'autant plus que la m\'ethode est assez complexe
et implique d'utiliser plusieurs signaux pour communiquer entre le widget
source et le widget d'arriv\'ee.\\

Pourtant, une fois le fonctionnement apr\'ehend\'e, la methode est
tr\`es efficace. C'est pour cela qu'elle \`a \'et\'e choisie pour
effectuer les trois op\'erations suivantes :\\


\subsection{Cr\'eation d'un patch}
On peut cr\'eer un patch en le prenant \`a la souris dans l'arbre
de la fen\^etre principale, et en le d\'eposant dans le fen\^etre
d'un patchwork. La liaison avec la fonction du noyau pour cr\'eer
un patch est faite, la cr\'eation d'un patch dans l'interface graphique
et dans le noyau est faite simultan\'ement.\\

La cr\'eation du widget d'un patch est faite via un fichier glade
qui contient ce qui est commun \`a tous les paches. Il faut ensuite
appliquer le nom du patch et cr\'eer les widgets correspondant aux
entr\'ees et aux sorties du patch cr\'e\'e.\\

\newpage

\subsection{D\'eplacement d'un patch}
Le d\'eplacement d'un patch dans le patchwork se fait lui aussi via un
glisser - d\'eposer depuis le nom du patch, en restant dans la fen\^etre
du patchwork. On r\'ecupere ainsi la nouvelle position du patch,
que l'on d\'eplace. On enregistre la position du patch car on en a besoin
pour dessiner les fl\`eches qui relient les patches entre eux.\\

\subsection{Connection des patches}
La connection d'une sortie d'un patch se fait en la tirant \`a la souris
jusqu'\`a une entr\'ee d'un autre patch. La connection v\'erifie
si les type de donn\'ees sont compatibles, si les patches ne sont pas
d\'ej\`a connect\'es et si tout est valide effectue la connection.
On enregistre alors la connection dans une liste cha\^in\'ee, pour
pouvoir afficher, detruire ou effectuer d'autres actions
sur les connections.\\


\section{Dessin des connections}

\section{Drag \& Drop}

\newpage
\section{Affichage}

\subsection{Le Patchwork}
\par
L'affichage du patchwork a chang\'e\ldots 
\\
En effet, un contexte graphique a \'et\'e ins\'er\'e. Il est cr\'e\'e grace \`a la lib cairo
et nous permet de dessiner, d'afficher des fen\^etres, et de pouvoir conna\^itre les 
caracteristiques de chacun par rapport aux autres. On peut toujour voir apparaitre 
la barre de contr\^ole permetant de demarrer, mettre en pause, arr\^eter un patchwork. 
Ces fonction sont limit\'ees \`a tester si les procedures des patches sont 
correctement execut\'ees.
\\

\subsection{Cairo et GTK}
\par
La derni\`ere version de Cairo, est ins\'er\'e\'e dans GTK. C'est \`a dire que la lib est 
directement g\'er\'ee par GTK. Mais afin de pouvoir profiter de ces avanc\'ees 
technologiques, il nous a fallu aqu\'erir la derni\`ere version de GTK. Les classiques 
'emerge' ou 'apt-get' de linux nous le permet, mais sous mac os x, c'est diff\'erent. 
En effet, l'installation de 'fink' (similaire a 'apt-get') n'a pas suffit, il nous 
a fallut installer 'darwinport' pour pouvoir b\'en\'eficier d'une version assez recente.
L'affichage est \`a present pr\'esent sur mac os x et linux. le probl\`eme \'etant resolu, 
on peut desormais jouir d'un bel affichage muti-os.
\newpage
\subsection{Les Patchs}
\par
Les patches sont enfin aparus ! H\'e oui, on peut voir maintenant appara\^itre les patchs 
et leurs contenus dans la fenetre du patchwork. On peut m\^eme la faire glisser de la 
liste des patches dans le patchwork gr\^ace au drag \& drop. 
\\
\par
La fen\^etre du patch est compos\'ee de plusieurs parties :

\begin{itemize}
\item Le titre
  \\
  Boite contenant le nom du patch.
\item Les entr\'ees / sorties
  \\
  Cr\'ees dynamiquement \`a la cr\'eation du patch dans le patchwork. Les checkbox 
  repr\'esentant les entr\'ees et les sorties sont li\'es \`a leur nom et une sortie 
  est reliable \`a une entr\'ee \`a la souris gr\^ace au drag \& drop.
\end{itemize}

\subsection{Les Pl\^eches}
\par
Les fl\`eches representent les liaisons entres les entr\'ees et les sorties des patches 
d'un patchwork. Elles sont dessin\'ees lors d'un d\'eplacement ou d'un drag \& drop. 
\\
\par
Afin de g\'erer l'affichage de toutes les fl\`eches, nous avons cr\'e\'e une liste 
de connections entre deux patches. Elle est mise \`a jour \`a chaque nouvelle 
connection et est parcourue quand on rafra\^ichi la fen\^etre du patchwork.
\\
\par
Le placement des points de d\'epart et d'arriv\'ee n'a pas \'et\'e une mince 
affaire. On a eu un petit probleme quant \`a la position de la checkbox dans sa boite, 
les 
insertions des checkbox \'etant dynamiques, on a du passer par des fonctions d'appel au parent et un certain 
d\'ecalage est venu s'ajouter pour pouvoir enfin avoir des fleches bien cadr\'es.

\section{Pour la prochaine soutenance}

Il reste juste \`a impl\'ementer dans l'interface la suppression
d\'une connection et la suppression d'un patch. Ce sont deux
fonctions assez triviales et ne devraient pas poser probl\`eme.\\

Si l'importation d'un fichier XML est termin\'ee pour la prochaine
soutenance, il faudra \'egalement ajouter la fonction dans l'interface
du patchwork\ldots\\

Pour prendre de l'avance il est possible que nous commencions \`a
inclure certaines fonctions du cluster dans l'interface graphique
comme nous l'avions fait \`a la premi\`ere soutenance avec la liste
des serveurs.\\
