\chapter{Introduction}

Nous voici \`a la seconde soutenance du projet Patchwork13, l'outil de 
synth\'ese modulaire !\\

Entre ces deux soutenance, le planning a \'et\'e respect\'e, tant au niveau du 
cluster qu'au niveau de l'interface, nous sommes m\^eme en avance sur les deux.
De plus des rajouts ont \'et\'e r\'ealis\'es dans le noyau permettant une 
gestion plus large des donn\'ees, ainsi qu'une exportation en xml.\\

Sur le cluster, bas\'es sur la base r\'ealis\'ee lors de la premi\`ere 
soutenance, nous avons implement\'e la quasi integralit\'e des messages de 
commande du server.\\

Concernant l'interface graphique, une fonction de drag and drop \`a \'et\'e 
concue pour permettre un affichage complet du patchwork que l'on veut cr\'eer.
De plus dans ce graphe les \'el\'ements sont r\'eellement reli\'es, ce n'est
pas qu'un simple affichage.\\

Au niveau de la SDL, des patchs de son et d\'affichage en openGL ont \'et\'e 
cr\'e\'es afin d'avoir des patchs visuels et auditifs utilisables et qui 
donneront une possibilit\'e de pr\'esenter quelque chose de plut\^ot 
impressionant pour les d\'emonstrations qui viendront aux futures soutenances.\\

Sur ce, passons tout de suite \`a la presentation plus approfondie de notre
travail.\\
